\documentclass{article}
\usepackage[utf8]{inputenc}
\usepackage{graphicx}
\usepackage{
    mathtools,
    amsfonts,
    amsmath,
    tikz-cd,
    amsthm,
    graphicx,
    amssymb,
    titlesec
}

\title{Notes on the Polyakov Measure, Moduli Space and String Amplitudes}
\author{Jake Bian}

\setcounter{secnumdepth}{4}
\titleformat{\paragraph}
{\normalfont\normalsize\bfseries}{\theparagraph}{1em}{}
\titlespacing*{\paragraph}
{0pt}{3.25ex plus 1ex minus .2ex}{1.5ex plus .2ex}

\begin{document}

    \graphicspath{ {images/} }
    \numberwithin{equation}{section}
    \newtheorem{defn}{Definition}[section]
    \newtheorem{prop}{Proposition}[section]
    \newtheorem{corr}{Corollary}[section]
    \newtheorem{remark}{Remark}[section]

    \maketitle

    \abstract{
        We give a pedagogical review of the moduli space measure in string theory and outline some aspects of string amplitudes. These notes are prepared for a  talk at UBC in Summer 2015 as a part of an ongoing string theory reading group. The choice of topics is heavily inspired by D'Hoker's 1999 IAS lectures \cite{ias99}.
    }

    \tableofcontents

    \section{Motivation \& Prerequisites}
        \subsection{Motivation}
    The Polyakov path integral is an integral over moduli space. The first five chapters of Polchinski's first volume \cite{polchinski} culminates to a measure for moduli space, known as the Polyakov Measure

    \begin{equation}
    \end{equation}

Rewriting the determinants as ghost fields, this looks like

    \begin{equation}
    \end{equation}


The goal of the first half of these notes is to gain as much geometrical intuition as possible about where these equations come from. The second half will be concerned with computing amplitudes.
\subsection{Differential Operators on Riemann Surfaces}
    \subsubsection{Basics}

Let's recall some basic notions from differential geometry on Riemann surfaces.

\paragraph{Riemann Surfaces}

    For our purposes, a Riemann surface $\Sigma$ is a 1-dimensional complex manifold.  We will be particularly interested in examples where $\Sigma$ is compact and connected, the topology of such $\Sigma$ is completely fixed by a genus $g$.

    In string theory we deal with Riemann surfaces equipped with metrics, i.e. Hermitian manifolds $(\Sigma, g)$. Being equivalent to a real Riemannian 2-manifold,  $(\Sigma, g)$ is conformally flat. That is, in each coordinate chart $U \subset \Sigma$, we have a diffeomophism:

    \begin{equation}
        \phi_U: U \to \mathbb C, ~~
        p \mapsto (z, \bar z)
    \end{equation}

    such that

    \begin{equation}
        g = k(z, \bar z) dz \otimes d\bar z
        \label{eq:riemann-surface-metric}
    \end{equation}

\paragraph{The Metric Isomorphism}

    As usual, $g$ gives an isomorphism $\tilde g$ between tangent and cotangent spaces at each point $p \in \Sigma$

        \begin{equation}
            \tilde g: T_p\Sigma \to T_p^*\Sigma ~,~ v \mapsto g(v, \cdot)
        \end{equation}

    As usual, it is convenient to view the inverse $g^{-1}$ of the $ g$ isomorphism as a rank $(2, 0)$ tensor defined in each coordinate chart by taking the matrix inverse of the components of $g$. In the case of Riemann surfaces, we have simply

        \begin{equation}
            g^{-1} = k^{-1}(z, \bar z) \partial_z \otimes \partial_{\bar z}
            \label{eq:2d-metric-inverse}
        \end{equation}

    These in turn give isomorphisms between tensors of different ranks
        \begin{equation}
            T_{[a,b]} \xrightarrow{g} T_{[a-1,b+1]}  \xrightarrow{g} ... \xrightarrow{g}  T_{[0,a + b]}
        \end{equation}

        \begin{equation}
            T_{[a,b]} \xrightarrow{g^{-1}} T_{[a+1,b-1]}  \xrightarrow{g^{-1}} ... \xrightarrow{g^{-1}}  T_{[a + b,0]}
        \end{equation}

    where $T_{[a,b]} \equiv (\otimes^a T\Sigma)(\otimes^b T^*\Sigma)$ is the space of rank $(a, b)$ tensors. These sequences of isomorphisms tell us we can restrict our attention to pure tensors without losing much generality.

\subsubsection{Line Bundles of Fixed-Weight Tensors}

    The most general rank $(0, w)$-tensor $T$ over $\Sigma$ lives in the tensor product cotangent bundle $\otimes^w T^*\Sigma$:

    \begin{equation}
        T = T{\mu_1 ... \mu_w} d\omega^{\mu_1}\otimes ... \otimes d\omega^{\mu_w}
    \end{equation}

    with $\mu_i \in \{0, 1\}$, $\omega^\mu \in \{z, \bar z\}$. This can always be written as

    \begin{equation}
        T = \tilde T(\otimes^m dz) (\otimes^n d \bar z)
        \label{eq:general-tensor}
    \end{equation}

    For some  $\tilde T \in \mathbb C$. and $m, n \in \mathbb Z^+$ with $m + n = w$. We say $T$ is a (contravariant) tensor of weight $(m, n)$, and conformal dimension $w$.

    We call the collection of these tensors $K_0^{(m n)}$

    \begin{equation}
        K_0^{(m, n)} \equiv \{ T \in \otimes^{m + n} T^*\Sigma ~|~ T \text{ has weight (m, n) }\}
    \end{equation}

    From \ref{eq:general-tensor} one can see that $K^{(m, n)}$ is a 1-dimesisonal vector space, hence $K$ defines a line bundle over $\Sigma$.

    One can make a trivial remark at this point: each space $K^{(m, n)}$ with different values of $(m,n)$ are isomorphic - namely they're all isomorphic to $\mathbb C$. One can then wonder how do these space differ. In our case, the only difference we will care about is the factor they gain under a pullback under a diffeomorphism. To this end, let's consider for a moment mixed tensors of the form

    \begin{equation}
        T = \tilde T(\otimes^m dz) (\otimes^n \partial \bar z) (\otimes^a \partial_z) (\otimes^b \partial_{\bar z})
    \end{equation}

    Observe that under a pullback by a diffeomorphism transform exactly like the tensor

    \begin{equation}
        \tilde T(\otimes^{(m-a)} dz) (\otimes^{(n-b)} d\bar z)
    \end{equation}

    Hence for our purposes, we will not distinguish between those two tensors. Let's give a name to this collection

    \begin{defn}

        $$
        K^{(m, n)} \equiv \{ \tilde T(\otimes^{m-a} dz) (\otimes^{n-b} \partial \bar z) (\otimes^a \partial_z)
        (\otimes^b \partial_{\bar z}) | a,b \in \mathbb Z^+, \tilde T \in \mathbb C \} / \sim
        $$

        where $\sim$ is the equivalence relation $A \sim B$ if they have the same scalar factor.
    \end{defn}

    The quotient by the identification of course makes sure that $K^{(m, n)}$ is isomorphic to $K_0^{(m,n)}$.


    Under the metric isomorphism discussed earlier, using the explicit form of the metric \ref{eq:2d-metric-inverse} (namely the fact that the diagonals vanish), we see that the the action of the metric is:

    \begin{equation}
        \tilde T(\otimes^m dz) (\otimes^n d \bar z) \mapsto \tilde T'(\otimes^m dz)(\otimes^{n-1} d \bar z)\otimes \partial_z
    \end{equation}

    The metric $g$ therefore gives isomorphisms

    \begin{equation}
        K^{(m, n)} \xrightarrow{g} K^{(m-1, n-1)}  \xrightarrow{g} ... \xrightarrow{g}  K^{(m-n, 0)}
    \end{equation}

    In particular, $K^{(m, n)} \simeq K^{(m-n, 0)}$. This tells us it is often sufficient to only care about tensors that only have z-indices. We therefore introduce the abbreviated notation:

    \begin{equation}
        K^{m} \equiv K^{(m, 0)}
    \end{equation}

    Note that in this notation we allow $m$ to be negative - a tensor of weight $- n$ for some $n > 0 $ is defined to be one of weight $(0, n)$.

    Being a complex line bundle over $\Sigma$, $K^m$ is a complex manifold in its own right. Furthermore we will equip it with an inner product $ \langle \cdot{} , \cdot{} \rangle$:

    \begin{equation}
        \langle \phi , \psi \rangle \equiv \int_\Sigma du ~g_{z \bar z}^{-m}~ \phi \psi
    \end{equation}

    where $du$ is the invariant measure on $(\Sigma, g)$.
\subsubsection{Differential Operators}

    For readability, let's slightly abuse our notation and use $K^{(m, n)}$ the space of sections over the line bundle which we previously called $K^{(m, n)}$.

    Let's consider a metric connection on $\Sigma$. The condition of metric compatibility reads:
    \begin{equation}
        \nabla_X g = 0 ~~ \forall X \text{ section to $T\Sigma$}
    \end{equation}

    In components, this translates to the demand that all components of $\nabla_a g_{bc}$ vanish. Using eq. \ref{eq:riemann-surface-metric}, this gives us all the Christoffel symbols. In the below we only state the results needed for our purposes, for the derivations, see for example \cite{nakahara}.

    The covariant derivative can be decomposed as:

    \begin{equation}
        \nabla \equiv \nabla_{\bar z} + \nabla_{ z}
    \end{equation}

    with

    \begin{equation}
        \nabla_{\bar z}: K^{(m, n)} \to K^{(m, n+1)}, ~~ \phi \mapsto \partial_{\bar z} \phi \otimes d \bar z
    \end{equation}

    \begin{equation}
        \nabla_z: K^{(m, n)} \to K^{(m + 1, n)}, ~~ \phi \mapsto (k^{-m} \partial_z k^m) \phi \otimes d z
    \end{equation}

    As mentioned, we would like to restrict our attention to pure z-index tensors, so that we can look at spaces $K^{a-b}$ as opposed to $K^{(a, b)}$. Setting $n=0$ in the domain of the operators above.

    Observe the latter of these two operators is already one which takes pure-z tensor to another pure-z one. To obtain from $\nabla_{\bar z}$ another operator which maps strictly to pure z-tensors, we can subsequently use the inverse metric to map $K^{(m, 1)} \to K^{m-1}$. That is, we define map:

    \begin{equation}
        \nabla^z \equiv g^{-1} \circ \nabla_{\bar z}
    \end{equation}

    observe this acts as:

    \begin{equation}
        \nabla^z: K^{(m, n)} \to K^{(m - 1, n)}, \phi \mapsto k^{-1}\partial_{\bar z} \phi \otimes \partial_{ z}
    \end{equation}

    From now on let's fix $n=0$ in the domain of these maps. For clarity of the below definitions we also add a superscript or subscript in brackets to denote the $z$-weight $m$ of the domain of these maps, e.g. $\nabla^z_{(m)}$, $\nabla_z^{(m)}$. Note that $\nabla_z$ and $\nabla^z$ are related by adjoints with respect to the metric inner product:

    \begin{equation}
        \nabla_z^\dagger = \nabla^z
    \end{equation}

    We can define Lapacians as compositions of these:

    \begin{equation}
        \triangle^+_m \equiv \nabla_z^{(m)} \nabla^z_{(m+1)} ~,~ K^{m+1} \to K^{m + 1}
    \end{equation}
    \begin{equation}
        \triangle^-_m \equiv \nabla^z_{(m)} \nabla_z^{(m-1)} ~,~ K^{m-1} \to K^{m - 1}
    \end{equation}

    All this can be summarized by a commutative diagram:

    \begin{equation}
        \begin{tikzcd}[row sep=4.5em]
            & K^{m-1} \\
            K^{m-1} \arrow{r}{\nabla_z}
            \arrow{ru}{\triangle^-} & K^m \arrow{u}{\nabla^z}
            \arrow{r}{\nabla_z} & K^{m+1}\\
            &K^{m+1} \arrow{ru}{\triangle^+} \arrow{u}{\nabla^z}
        \end{tikzcd}
    \end{equation}






    \section{Moduli Space}
        \subsection{Motivation}
    The Polyakov path integral is an integral over moduli space. The first five chapters of Polchinski's first volume \cite{polchinski} culminates to a measure for moduli space, known as the Polyakov Measure

    \begin{equation}
    \end{equation}

Rewriting the determinants as ghost fields, this looks like

    \begin{equation}
    \end{equation}


The goal of the first half of these notes is to gain as much geometrical intuition as possible about where these equations come from. The second half will be concerned with computing amplitudes.
\subsection{Differential Operators on Riemann Surfaces}
    \subsubsection{Basics}

Let's recall some basic notions from differential geometry on Riemann surfaces.

\paragraph{Riemann Surfaces}

    For our purposes, a Riemann surface $\Sigma$ is a 1-dimensional complex manifold.  We will be particularly interested in examples where $\Sigma$ is compact and connected, the topology of such $\Sigma$ is completely fixed by a genus $g$.

    In string theory we deal with Riemann surfaces equipped with metrics, i.e. Hermitian manifolds $(\Sigma, g)$. Being equivalent to a real Riemannian 2-manifold,  $(\Sigma, g)$ is conformally flat. That is, in each coordinate chart $U \subset \Sigma$, we have a diffeomophism:

    \begin{equation}
        \phi_U: U \to \mathbb C, ~~
        p \mapsto (z, \bar z)
    \end{equation}

    such that

    \begin{equation}
        g = k(z, \bar z) dz \otimes d\bar z
        \label{eq:riemann-surface-metric}
    \end{equation}

\paragraph{The Metric Isomorphism}

    As usual, $g$ gives an isomorphism $\tilde g$ between tangent and cotangent spaces at each point $p \in \Sigma$

        \begin{equation}
            \tilde g: T_p\Sigma \to T_p^*\Sigma ~,~ v \mapsto g(v, \cdot)
        \end{equation}

    As usual, it is convenient to view the inverse $g^{-1}$ of the $ g$ isomorphism as a rank $(2, 0)$ tensor defined in each coordinate chart by taking the matrix inverse of the components of $g$. In the case of Riemann surfaces, we have simply

        \begin{equation}
            g^{-1} = k^{-1}(z, \bar z) \partial_z \otimes \partial_{\bar z}
            \label{eq:2d-metric-inverse}
        \end{equation}

    These in turn give isomorphisms between tensors of different ranks
        \begin{equation}
            T_{[a,b]} \xrightarrow{g} T_{[a-1,b+1]}  \xrightarrow{g} ... \xrightarrow{g}  T_{[0,a + b]}
        \end{equation}

        \begin{equation}
            T_{[a,b]} \xrightarrow{g^{-1}} T_{[a+1,b-1]}  \xrightarrow{g^{-1}} ... \xrightarrow{g^{-1}}  T_{[a + b,0]}
        \end{equation}

    where $T_{[a,b]} \equiv (\otimes^a T\Sigma)(\otimes^b T^*\Sigma)$ is the space of rank $(a, b)$ tensors. These sequences of isomorphisms tell us we can restrict our attention to pure tensors without losing much generality.

\subsubsection{Line Bundles of Fixed-Weight Tensors}

    The most general rank $(0, w)$-tensor $T$ over $\Sigma$ lives in the tensor product cotangent bundle $\otimes^w T^*\Sigma$:

    \begin{equation}
        T = T{\mu_1 ... \mu_w} d\omega^{\mu_1}\otimes ... \otimes d\omega^{\mu_w}
    \end{equation}

    with $\mu_i \in \{0, 1\}$, $\omega^\mu \in \{z, \bar z\}$. This can always be written as

    \begin{equation}
        T = \tilde T(\otimes^m dz) (\otimes^n d \bar z)
        \label{eq:general-tensor}
    \end{equation}

    For some  $\tilde T \in \mathbb C$. and $m, n \in \mathbb Z^+$ with $m + n = w$. We say $T$ is a (contravariant) tensor of weight $(m, n)$, and conformal dimension $w$.

    We call the collection of these tensors $K_0^{(m n)}$

    \begin{equation}
        K_0^{(m, n)} \equiv \{ T \in \otimes^{m + n} T^*\Sigma ~|~ T \text{ has weight (m, n) }\}
    \end{equation}

    From \ref{eq:general-tensor} one can see that $K^{(m, n)}$ is a 1-dimesisonal vector space, hence $K$ defines a line bundle over $\Sigma$.

    One can make a trivial remark at this point: each space $K^{(m, n)}$ with different values of $(m,n)$ are isomorphic - namely they're all isomorphic to $\mathbb C$. One can then wonder how do these space differ. In our case, the only difference we will care about is the factor they gain under a pullback under a diffeomorphism. To this end, let's consider for a moment mixed tensors of the form

    \begin{equation}
        T = \tilde T(\otimes^m dz) (\otimes^n \partial \bar z) (\otimes^a \partial_z) (\otimes^b \partial_{\bar z})
    \end{equation}

    Observe that under a pullback by a diffeomorphism transform exactly like the tensor

    \begin{equation}
        \tilde T(\otimes^{(m-a)} dz) (\otimes^{(n-b)} d\bar z)
    \end{equation}

    Hence for our purposes, we will not distinguish between those two tensors. Let's give a name to this collection

    \begin{defn}

        $$
        K^{(m, n)} \equiv \{ \tilde T(\otimes^{m-a} dz) (\otimes^{n-b} \partial \bar z) (\otimes^a \partial_z)
        (\otimes^b \partial_{\bar z}) | a,b \in \mathbb Z^+, \tilde T \in \mathbb C \} / \sim
        $$

        where $\sim$ is the equivalence relation $A \sim B$ if they have the same scalar factor.
    \end{defn}

    The quotient by the identification of course makes sure that $K^{(m, n)}$ is isomorphic to $K_0^{(m,n)}$.


    Under the metric isomorphism discussed earlier, using the explicit form of the metric \ref{eq:2d-metric-inverse} (namely the fact that the diagonals vanish), we see that the the action of the metric is:

    \begin{equation}
        \tilde T(\otimes^m dz) (\otimes^n d \bar z) \mapsto \tilde T'(\otimes^m dz)(\otimes^{n-1} d \bar z)\otimes \partial_z
    \end{equation}

    The metric $g$ therefore gives isomorphisms

    \begin{equation}
        K^{(m, n)} \xrightarrow{g} K^{(m-1, n-1)}  \xrightarrow{g} ... \xrightarrow{g}  K^{(m-n, 0)}
    \end{equation}

    In particular, $K^{(m, n)} \simeq K^{(m-n, 0)}$. This tells us it is often sufficient to only care about tensors that only have z-indices. We therefore introduce the abbreviated notation:

    \begin{equation}
        K^{m} \equiv K^{(m, 0)}
    \end{equation}

    Note that in this notation we allow $m$ to be negative - a tensor of weight $- n$ for some $n > 0 $ is defined to be one of weight $(0, n)$.

    Being a complex line bundle over $\Sigma$, $K^m$ is a complex manifold in its own right. Furthermore we will equip it with an inner product $ \langle \cdot{} , \cdot{} \rangle$:

    \begin{equation}
        \langle \phi , \psi \rangle \equiv \int_\Sigma du ~g_{z \bar z}^{-m}~ \phi \psi
    \end{equation}

    where $du$ is the invariant measure on $(\Sigma, g)$.
\subsubsection{Differential Operators}

    For readability, let's slightly abuse our notation and use $K^{(m, n)}$ the space of sections over the line bundle which we previously called $K^{(m, n)}$.

    Let's consider a metric connection on $\Sigma$. The condition of metric compatibility reads:
    \begin{equation}
        \nabla_X g = 0 ~~ \forall X \text{ section to $T\Sigma$}
    \end{equation}

    In components, this translates to the demand that all components of $\nabla_a g_{bc}$ vanish. Using eq. \ref{eq:riemann-surface-metric}, this gives us all the Christoffel symbols. In the below we only state the results needed for our purposes, for the derivations, see for example \cite{nakahara}.

    The covariant derivative can be decomposed as:

    \begin{equation}
        \nabla \equiv \nabla_{\bar z} + \nabla_{ z}
    \end{equation}

    with

    \begin{equation}
        \nabla_{\bar z}: K^{(m, n)} \to K^{(m, n+1)}, ~~ \phi \mapsto \partial_{\bar z} \phi \otimes d \bar z
    \end{equation}

    \begin{equation}
        \nabla_z: K^{(m, n)} \to K^{(m + 1, n)}, ~~ \phi \mapsto (k^{-m} \partial_z k^m) \phi \otimes d z
    \end{equation}

    As mentioned, we would like to restrict our attention to pure z-index tensors, so that we can look at spaces $K^{a-b}$ as opposed to $K^{(a, b)}$. Setting $n=0$ in the domain of the operators above.

    Observe the latter of these two operators is already one which takes pure-z tensor to another pure-z one. To obtain from $\nabla_{\bar z}$ another operator which maps strictly to pure z-tensors, we can subsequently use the inverse metric to map $K^{(m, 1)} \to K^{m-1}$. That is, we define map:

    \begin{equation}
        \nabla^z \equiv g^{-1} \circ \nabla_{\bar z}
    \end{equation}

    observe this acts as:

    \begin{equation}
        \nabla^z: K^{(m, n)} \to K^{(m - 1, n)}, \phi \mapsto k^{-1}\partial_{\bar z} \phi \otimes \partial_{ z}
    \end{equation}

    From now on let's fix $n=0$ in the domain of these maps. For clarity of the below definitions we also add a superscript or subscript in brackets to denote the $z$-weight $m$ of the domain of these maps, e.g. $\nabla^z_{(m)}$, $\nabla_z^{(m)}$. Note that $\nabla_z$ and $\nabla^z$ are related by adjoints with respect to the metric inner product:

    \begin{equation}
        \nabla_z^\dagger = \nabla^z
    \end{equation}

    We can define Lapacians as compositions of these:

    \begin{equation}
        \triangle^+_m \equiv \nabla_z^{(m)} \nabla^z_{(m+1)} ~,~ K^{m+1} \to K^{m + 1}
    \end{equation}
    \begin{equation}
        \triangle^-_m \equiv \nabla^z_{(m)} \nabla_z^{(m-1)} ~,~ K^{m-1} \to K^{m - 1}
    \end{equation}

    All this can be summarized by a commutative diagram:

    \begin{equation}
        \begin{tikzcd}[row sep=4.5em]
            & K^{m-1} \\
            K^{m-1} \arrow{r}{\nabla_z}
            \arrow{ru}{\triangle^-} & K^m \arrow{u}{\nabla^z}
            \arrow{r}{\nabla_z} & K^{m+1}\\
            &K^{m+1} \arrow{ru}{\triangle^+} \arrow{u}{\nabla^z}
        \end{tikzcd}
    \end{equation}






    \section{String Amplitudes}
        \subsection{Tree Level}
        \subsection{Witten's Geometric Intepretation of $i \epsilon$}
            Here we \textbf{briefly} describe a geometric interpretation of the $i \epsilon$ prescription in String theory due to Witten \cite{witten-ieps}.
        \subsection{Beyond Tree Level}



    \bibliography{refs}{}
    \bibliographystyle{abbrv}

\end{document}
