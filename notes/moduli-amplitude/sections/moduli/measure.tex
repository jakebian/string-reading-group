\subsubsection{The Setup}
    Armed with the decomposition derived above, we can write the most general $\omega \in T_gMet(g)$ as

    \begin{equation}
        \omega = \varphi dz \otimes d\bar z + (\nabla_z V + \rho) dz \otimes d z + (\nabla_{\bar z} V + \bar \rho) d\bar z \otimes d \bar z
    \end{equation}

    We define a measure $\mathcal D \omega$ by

    \begin{equation}
        \int_{T_g Met(g)} \mathcal D \omega e^{\frac12 |\omega|^2} = 1
    \end{equation}


    The norm that appears in the exponential is evaluated using our metric inner product. Using the decomposition this reads

    \begin{equation}
        |\omega|^2 = |\varphi|^2 + (|\nabla_z V|^2 + |\rho|^2 + cc.)
    \end{equation}

    where we used the orthogonality of $\rho$ and $\nabla_z V$.


    The decomposition used here is orthogonal, this implies we can write

    \begin{equation}
        \mathcal D \omega = \mathcal D \phi \mathcal
        ~ D (\nabla_z V)\mathcal
        ~ D (\nabla_{\bar z} V)
        ~ \mathcal D \rho \mathcal D \bar \rho
    \end{equation}

    with each term in the product a measure defined by a Gaussian:


    \begin{equation}
         \int_{T_g Met(g)} \mathcal D \phi ~ e^{\frac12 |\phi|^2} = 1
    \end{equation}
    \begin{equation}
         \int_{T_g Met(g)} \mathcal D (\nabla_z V) ~ e^{\frac12 |(\nabla_z V)|^2} = 1
    \end{equation}
    \begin{equation}
         \int_{T_g Met(g)} \mathcal D \rho ~ e^{\frac12 |\rho|^2} = 1
    \end{equation}


    where the complex conjugate counterparts are similarly defined. The Jacobian $J_\omega$ from $\mathcal D \omega$ is then a product of the Jacobians from these measures:

    \begin{equation}
        J_\omega = J_\phi |J_{\nabla_z V}|^2 |J_\rho|^2
    \end{equation}


\subsubsection{The Gauge Orbit Measures}

    Let's evaluate these measures one by one. First one is easy:

    \begin{align}
         \int_{T_g Met(g)} \mathcal D \phi e^{\frac12 |\phi|^2} &= 1 \\
            \implies J_\phi &= 1
    \end{align}

    Now onto the $D (\nabla_z V)$ part. We observe

    \begin{equation}
        |\nabla_z V|^2 = \langle \nabla_z V, \nabla_z V \rangle = \langle V, \triangle^- V \rangle
    \end{equation}

    This form makes the Gaussian integral easy to compute

    \begin{align}
        J_{\nabla_z V}^{-1} &= \int_{T_g Met(g)} \mathcal D (\nabla_z V) ~ e^{\frac12 |(\nabla_z V)|^2}\\
        &= \int_{T_g Met(g)} \mathcal D (\nabla_z V) ~ e^{\frac12 \langle V, \triangle^- V \rangle}\\
        &= det(\triangle^-)^{-\frac12}
    \end{align}

    Multiplying by the obviously analogous result for the complex conjugate counterpart, we see

    \begin{equation}
        |J_{\nabla_z V}|^2 = det(\triangle^-)
    \end{equation}

\subsubsection{The Gauge Slice Measure}


We can now consider a basis $(\mu_j)_{1 \le j \le dim(Mod(\Sigma))}$ for $Ker(\nabla^{(2)}_{\bar z})$. Such bases (and their complex conjugates) are called \textit{Beltrami Differentials}. 
