\subsubsection{The Setup}
    Armed with the decomposition derived above, we can write the most general $\omega \in T_gMet(g)$ as

    \begin{equation}
        \omega = \varphi dz \otimes d\bar z + (\nabla_z V + \rho) dz \otimes d z + (\nabla_{\bar z} V + \bar \rho) d\bar z \otimes d \bar z
    \end{equation}

    We define a measure $\mathcal D \omega$ by

    \begin{equation}
        \int_{T_g Met(g)} \mathcal D \omega e^{\frac12 |\omega|^2} = 1
    \end{equation}


    The norm that appears in the exponential is evaluated using our metric inner product. Using the decomposition this reads

    \begin{equation}
        |\omega|^2 = |\varphi|^2 + (|\nabla_z V|^2 + |\rho|^2 + cc.)
    \end{equation}

    where we used the orthogonality of $\rho$ and $\nabla_z V$.


    The decomposition used here is orthogonal, this implies we can write

    \begin{equation}
        \mathcal D \omega = \mathcal D \phi \mathcal
        ~ D (\nabla_z V)\mathcal
        ~ D (\nabla_{\bar z} V)
        ~ \mathcal D \rho \mathcal D \bar \rho
    \end{equation}

    with each term in the product a measure defined by a Gaussian:


    \begin{equation}
         \int_{T_g Met(g)} \mathcal D \phi ~ e^{\frac12 |\phi|^2} = 1
    \end{equation}
    \begin{equation}
         \int_{T_g Met(g)} \mathcal D (\nabla_z V) ~ e^{\frac12 |(\nabla_z V)|^2} = 1
    \end{equation}
    \begin{equation}
         \int_{T_g Met(g)} \mathcal D \rho ~ e^{\frac12 |\rho|^2} = 1
    \end{equation}


    where the complex conjugate counterparts are similarly defined. The Jacobian $J_\omega$ from $\mathcal D \omega$ is then a product of the Jacobians from these measures:

    \begin{equation}
        J_\omega = J_\phi |J_{\nabla_z V}|^2 |J_\rho|^2
    \end{equation}


\subsubsection{The Gauge Orbit Measures}

    Let's evaluate these measures one by one. First one is easy:

    \begin{align}
         \int_{T_g Met(g)} \mathcal D \phi e^{\frac12 |\phi|^2} &= 1 \\
            \implies J_\phi &= 1
    \end{align}

    Now onto the $D (\nabla_z V)$ part. We observe

    \begin{equation}
        |\nabla_z V|^2 = \langle \nabla_z V, \nabla_z V \rangle = \langle V, \triangle^- V \rangle
    \end{equation}

    This form makes the Gaussian integral easy to compute

    \begin{align}
        J_{\nabla_z V}^{-1} &= \int_{T_g Met(g)} \mathcal D (\nabla_z V) ~ e^{\frac12 |(\nabla_z V)|^2}\\
        &= \int_{T_g Met(g)} \mathcal D (\nabla_z V) ~ e^{\frac12 \langle V, \triangle^- V \rangle}\\
        &= det(\triangle^-)^{-\frac12}
    \end{align}

    Multiplying by the obviously analogous result for the complex conjugate counterpart, we see

    \begin{equation}
        |J_{\nabla_z V}|^2 = det(\triangle^-)
    \end{equation}

\subsubsection{The Gauge Slice Measure}

    Note that so far, where the components of $\omega$ tangent to the gauge orbit have been explicit, the component $\rho$ tangent to the gauge orbit have been symbolic - namely $\rho$ is an element in the complex vector space $ker \nabla^{(2)}_{\bar z}$. To find the $\rho$ measure, we must choose a basis for $ker \nabla^{(2)}_{\bar z}$, then integrate the components of $\rho$ in this basis. Additionally, we would like this basis to be Weyl invariant.  Elements $(\mu_j)_{1 \le j \le dim(Mod(\Sigma))}$ in a Weyl-invariant basis for $ker \nabla^{(2)}_{\bar z}$ are called \textit{Beltrami Differentials}.

    In terms of the Beltrami Differentials, we can write $\rho$ and $\bar \rho$ as

    \begin{align}
        \rho & = \rho^j \mu_j\\
        \bar \rho & = \bar \rho^j \bar \mu_j\\
    \end{align}

    Since the Beltrami differentials are in general not an orthogonal basis, let's write the above in an orthogonal basis. Let $(\phi_j)_{1 \le j \le dim(Mod(\Sigma))}$ be an orthogonal basis for $ker \nabla^{(2)}_{\bar z}$. Freshman linear algebra tells us how to write $\rho$ in this basis

    \begin{equation}
        \rho = \sum_j  \frac 1 {|\phi_j|} \rho^k \langle \phi_j , \mu_k \rangle \phi_j
    \end{equation}

    Taking a inner product of this with itself, we get

    \begin{equation}
        |\rho|^2 = \sum_j  \rho^l \rho^k \frac 1 {|\phi_j|^2} \langle \phi_j , \mu_k \rangle \langle \phi_j , \mu_l \rangle
    \end{equation}

    We can now compute the Gaussian

    \begin{align}
        J_\rho^{-1} &= \int e^{\frac12 |\rho|^2} \\
            &= e^{\rho^l \sum_j \frac 1 {|\phi_j|^2} \langle \phi_j , \mu_k \rangle \langle \phi_j , \mu_l \rangle \rho^k }\\
            &= det\left(\sum_j \frac 1 {|\phi_j|^2} \langle \phi_j , \mu_k \rangle \langle \phi_j , \mu_l \rangle \right)^{-\frac12}\\
            &= \left(\frac {det(\langle \phi_j , \mu_k \rangle)^2} {det(\langle \phi_j, \phi_k \rangle)} \right)^{-\frac12}
    \end{align}

    Multiplying with the complex conjugate result, we find

    \begin{equation}
        |J_\rho|^2 = \frac { | det(\langle \phi_j , \mu_k \rangle) | ^2} {det(\langle \phi_j, \phi_k \rangle)}
    \end{equation}

\subsubsection{Putting it together}
    Combining our Jacobians, we find

    \begin{equation}
        J_\omega = J_\phi |J_{\nabla_z V}|^2 |J_\rho|^2 = det(\triangle^-) \frac { | det(\langle \phi_j , \mu_k \rangle) | ^2} {det(\langle \phi_j, \phi_k \rangle)}
    \end{equation}

    Putting this Jacobian in the $Met(g)$ metric decomposition \label{eq:measure-decomp}, we get

    \begin{equation}
        \mathcal D g = J \mathcal D m \mathcal D \sigma \mathcal D v =  det(\triangle^-) \frac { | det(\langle \phi_j , \mu_k \rangle) | ^2} {det(\langle \phi_j, \phi_k \rangle)} \mathcal D m \mathcal D \sigma \mathcal D v
    \end{equation}

    We obtained a measure for $Met(g)$ decomposed exactly like we proposed at the beginning. In fact we can be even more explicit. Let $(m_j, \bar m_j)$ be a basis for $Mod(\Sigma)$, then we can write:

    \begin{equation}
        \mathcal Dm = \prod_j dm_j d\bar m_j
    \end{equation}

    We now make one final modification which will seem un-motivated at this point. The motivation will only become clear in a discussion of the Weyl transformation property of the measure in addition to an application of the Fadeev-Poppov procedure. For now we only perform the following re-writing for the sake of completeness.

    Let $\psi_a$ be an orthogonal basis for $Ker \nabla_{\bar z^(1)}$, define:

    \begin{equation}
        Z^-_{(-1)} (g) \equiv  \frac{det\triangle^-}{det\langle \phi_j, \phi_k\rangle det\langle \psi_a, \psi_b \rangle}
    \end{equation}

    $Z^-_{(-1)}$ will turn out to be the Fadeev-Poppov determinant. We also rescale our $Diff$ orbit measure to cancel out the $\psi$ factor we added

    \begin{equation}
        D'v = det\langle \psi_a, \psi_b \rangle Dv
    \end{equation}

    Then our measure is rewritten as

    \begin{equation}
        \mathcal D g = \mathcal D \sigma \mathcal D' v Z^-_{(-1)}  | det\langle \phi_j , \mu_k \rangle | ^2 \prod_j dm_j d\bar m_j
    \end{equation}

    This is what we will call the Polyakov measure.