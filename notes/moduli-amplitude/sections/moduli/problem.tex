Let $\Sigma$ be a Riemann surface. Let $Met(\Sigma)$ denote the space of all metrics on $\Sigma$. Define the moduli space as the quotient

\begin{equation}
    Mod(\Sigma) \equiv Met(\Sigma)/(Diff(\Sigma) \ltimes Weyl(\Sigma))
\end{equation}

where $\ltimes$ denotes a semidirect product. Viewing this quotient space as a local product $Met(\Sigma) \xrightarrow{\pi} Mod(\Sigma)$, then locally, a metric $g \in Met(\Sigma)$ can be written as

\begin{equation}
    g \sim (h, \sigma)
\end{equation}

for $h \in Mod(\Sigma)$, and $\sigma \in Diff(\Sigma) \ltimes Weyl(\Sigma)$. In physicist terms, $h$ lives in a gauge slice and $\sigma$ live in a gauge orbit.


Observe now on $Met(\Sigma)$ we can define a natural gaussian measure $\mathcal{D}g$ with respect to any tangent space $T_g Met(\Sigma)$. Our goal is then to find a integration measure on $Mod(\Sigma)$ by "projecting" the $Met(\Sigma)$ measure $\mathcal D g$. More concretely, we want to decompose $\mathcal D g$ as

\begin{equation}
    \mathcal D g = J \mathcal D h \mathcal D \sigma
\end{equation}

for some Jacobian $J$, a measure $\mathcal D h$ for $Mod(\Sigma)$, and a measure $\mathcal D \sigma$ for $\pi^{-1}(h)$. We can subsequently perform the integration over the gauge orbits to obtain the measure on moduli space.

By this decmposition, it is clear that the main task at hand is to evaluate the Jacobian $J$. Evaluating $J$ at $g \in Met(\Sigma)$ is of course the same thing as evaluating the Jacobian from a basis change at $T_g Met(\Sigma)$. Our strategy therefore is to decompose the tangent space $T_g Met(\Sigma)$.
