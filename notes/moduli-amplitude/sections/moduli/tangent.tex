Let $g \in Met(\Sigma)$. Physically, one can think of the tangent space $T_g Met(\Sigma)$ as variations of $g$

\begin{equation}
    \delta g \equiv \varepsilon \omega~,~ \omega \in T_g Met(\Sigma)
\end{equation}


$T_g Met(\Sigma)$ is a space of rank $(0, 2)$ tensors, which can be in turn decomposed by weights. Using notation introduced earlier

\begin{equation}
    T_g Met(\Sigma) \simeq T_{[0, 2]} \simeq T^{(1, 1)} \oplus T^{(0, 2)} \oplus T^{(2, 0)}
\end{equation}

We examine each space on the RHS one by one.

\subsubsection{$T^{(1, 1)}$}

    The most general element in $T^{(1, 1)}$ can be written as

    \begin{equation}
        \omega = \varphi dz \otimes d\bar z
    \end{equation}

    with $\varphi \in \mathbb C$. Now consider the metric perturbation

    \begin{equation}
        g \mapsto g + \delta g
    \end{equation}

    In isothermal coordinates and notation from before

    \begin{equation}
        k dz \otimes d\bar z \mapsto k dz \otimes d\bar z + \delta g
    \end{equation}

    Now consider a perturbation $\delta g = \epsilon \omega$ with $\omega \in T^{(1, 1)}$, the change is

    \begin{equation}
        k dz \otimes d\bar z \mapsto k dz \otimes d\bar z +  \varepsilon \varphi dz \otimes d\bar z
    \end{equation}


    This is simply a scaling of the metric. In other words, this can be written as

    \begin{equation}
        g \mapsto f_\varepsilon^* (g)
    \end{equation}

    for some map $f_\varepsilon \in Weyl(\Sigma)$. What we have just shown is the following: every $\omega \in T^{(1, 1)}$ is a fundamental tensor field generated by some element in the Lie algebra of $Weyl(\Sigma)$. Conversely, every element in the Lie algebra of $Weyl(\Sigma)$ generates a vector field which evaluates to some $\omega \in T^{(1, 1)}$.

    To summarize this colloquially: $T^{(1,1)}$ is covered by the infinitesmall action of the Weyl group.

\subsubsection{$T^{(2, 0)}$ and $T^{(0, 2)}$}

    Inspired by how we were able to identify $T^{(1,1)}$ with pullbacks of Weyl maps on the metric, let's consider the action of a diffeomorphism on the metric. Any diffeomorphism connected to the identity $f \in Diff_0(\Sigma)$ can be obtained as the flow generated by a vector field $V$ in $T\Sigma$. The infinitesmall action of $f_\epsilon$ on the metric can be read off from the definition of the Lie derivative

    \begin{equation}
        \delta g = f_\epsilon^*(g) - g = \epsilon \mathcal L_V g
    \end{equation}

    In some local coordinate chart, the Lie derivative of the metric can be nicely expressed in terms of the covariant derivative:

    \begin{equation}
        \delta g_{ab} = (\mathcal L_V g)_{ab} = \nabla_a V_b + \nabla_b V_a
    \end{equation}

    where the vector indices are lowered by the metric isomorphism. By the discussion in the previous section, we know the off diagonal components correspond to Weyl scalings. Let's therefore restrict ourselves to considering "diagonal variations" $\delta_d g$ where we set the off diagonal entries of the variation to $0$, the result is then

    \begin{equation}
        \label{eq:diagonal-variation}
        \frac{1}{2} \delta_d g =  \nabla_z V_z dz \otimes dz + \nabla_{\bar z} V_{\bar z} d\bar z \otimes d\bar z
    \end{equation}

    This expression tells us

    \begin{equation}
        \delta_d g \in Im(\nabla_z^{(1)}) \oplus Im(\nabla_{\bar z}^{(1)})
        \subset T^{(2,0)} \oplus T^{(0,2)}
    \end{equation}

    We now show that $Im(\nabla_z^{(1)}) \oplus Im(\nabla_{\bar z}^{(1)})$ is a strict subset of $T^{(2,0)} \oplus T^{(0,2)}$, and give explicitly the orthogonal complement of the former in the latter. To appreciate this result that we're going to prove shortly, let's take a step back and look at the geometric picture of all this.

    For readability let's denote

    \begin{align}
        D &\equiv Im(\nabla_z^{(1)}) \oplus Im(\nabla_{\bar z}^{(1)}) \\
        \bar D &\equiv \text{orthogonal complement of } D \text{ in } T^{(2,0)} \oplus T^{(0,2)}
    \end{align}

    Now:

    \begin{itemize}
        \item We just saw $D$ is generated by pullbacks of diffeomorphisms on $\Sigma$. In other words $D$ is tangent to the gauge orbit.
        \item The orthogonal complement $\bar D$ is therefore tangent to the gauge slice.
        \item We saw in the previous section that all of $T^{(1,1)}$ is generated by Weyl maps, therefore all of $T^{(1,1)}$ is tangent to the gauge orbit.
        \item This leaves $\bar D$ as exactly the subspace of $T_g Met(\Sigma)$ that is tangent to the gauge orbit.
    \end{itemize}


    Our task therefore has been reduced to finding $\bar D$. Since $T^{(2, 0)}$ and $T^{(0, 2)}$ are related by complex conjugation, it is sufficient to consider $T^{(2, 0)}$. The crucial result is the following

    \begin{prop}
        $T^{(2,0)} \simeq Im(\nabla^{(1)}_z) \oplus Ker(\nabla^{(2)}_{\bar z}) \oplus CKV$
    \end{prop}

    \begin{proof}
    \end{proof}


    To simplify the following arguments, for now let's ignore the $CKV$ part (note they vanish for genus $\ge 2$ anyway) and come back to them. To summarize everything in this section, omitting the $CKV$ subspace, what we have found is

    \begin{prop}
        $T_g Met(\Sigma) = [ T^{(1,1)} \oplus Im(\nabla^{(1)}_z) ] \oplus Ker(\nabla^{(2)}_{\bar z})\oplus cc.$
    \end{prop}

    As argued, the subspace in the square brackets (and its complex conjugate counterparts) are tangent to the gauge orbit, while the space of holomorphic differentials $Ker(\nabla^{(2)}_{\bar z})\oplus Ker(\nabla^{(2)}_{z})$ are directions along the gauge slice.

    We can now consider a basis $(\mu_j)_{1 \le j \le dim(Mod(\Sigma))}$ for $Ker(\nabla^{(2)}_{\bar z})$. Such bases (and their complex conjugates) are called \textit{Beltrami Differentials}.
