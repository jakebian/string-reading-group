Let's recall some basic notions from differential geometry on Riemann surfaces.

\paragraph{Riemann Surfaces}

    For our purposes, a Riemann surface $\Sigma$ is a 1-dimensional complex manifold.  We will be particularly interested in examples where $\Sigma$ is compact and connected, the topology of such $\Sigma$ is completely fixed by a genus $g$.

    In string theory we deal with Riemann surfaces equipped with metrics, i.e. Hermitian manifolds $(\Sigma, g)$. Being equivalent to a real Riemannian 2-manifold,  $(\Sigma, g)$ is conformally flat. That is, in each coordinate chart $U \subset \Sigma$, we have a diffeomophism:

    \begin{equation}
        \phi_U: U \to \mathbb C, ~~
        p \mapsto (z, \bar z)
    \end{equation}

    such that

    \begin{equation}
        g = k(z, \bar z) dz \otimes d\bar z
        \label{eq:riemann-surface-metric}
    \end{equation}

\paragraph{The Metric Isomorphism}

    As usual, $g$ gives an isomorphism $\tilde g$ between tangent and cotangent spaces at each point $p \in \Sigma$

        \begin{equation}
            \tilde g: T_p\Sigma \to T_p^*\Sigma ~,~ v \mapsto g(v, \cdot)
        \end{equation}

    As usual, it is convenient to view the inverse $g^{-1}$ of the $ g$ isomorphism as a rank $(2, 0)$ tensor defined in each coordinate chart by taking the matrix inverse of the components of $g$. In the case of Riemann surfaces, we have simply

        \begin{equation}
            g^{-1} = k^{-1}(z, \bar z) \partial_z \otimes \partial_{\bar z}
            \label{eq:2d-metric-inverse}
        \end{equation}

    These in turn give isomorphisms between tensors of different ranks
        \begin{equation}
            T_{[a,b]} \xrightarrow{g} T_{[a-1,b+1]}  \xrightarrow{g} ... \xrightarrow{g}  T_{[0,a + b]}
        \end{equation}

        \begin{equation}
            T_{[a,b]} \xrightarrow{g^{-1}} T_{[a+1,b-1]}  \xrightarrow{g^{-1}} ... \xrightarrow{g^{-1}}  T_{[a + b,0]}
        \end{equation}

    where $T_{[a,b]} \equiv (\otimes^a T\Sigma)(\otimes^b T^*\Sigma)$ is the space of rank $(a, b)$ tensors. These sequences of isomorphisms tell us we can restrict our attention to pure tensors without losing much generality.
